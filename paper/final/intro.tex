\chapter*{\large ВВЕДЕНИЕ}
\addcontentsline{toc}{chapter}{ВВЕДЕНИЕ}
В современном мире развитие нейронных сетей и глубинного обучения происходит очень быстро. Регулярно появляются новые методы обучения нейронный сетей,
 новые архитектуры, новые слои или трюки, каждые из которых имеют свои особенные полезные эффекты и поведение. К сожалению, из-за многомерности данных, их сложной структуры и общей
 сложности задач обучения нейронных сетей, люди всё меньше и меньше могут объяснить, почему тот или иной трюк, слой работают. В данном случае речь не идёт об
 интерпретации нейронных сетей, а о поиске строгих математических доказательств и обоснований их работы. Наши открытия в этой области обгоняют
 нашу способность их постичь. Позднее некоторые исследования в этой области и полученные методы получили название --- Нейробайесовские методы.

Получение истинного понимания принципа работы нейронных сетей позволяют усовершенствовать уже существующие методы и изобрести новые. По этой теме сделано
 большое количество исследований и работ на английском языке. На русском языке популяризатором этих методов стал Ветров Дмитрий Петрович, который ведёт исследования в этой области.

Целью курсовой работы будет разобрать самые базовые концепции и понятия нейробайесовских методов, собрать их в одной работе и предоставить весь материал, чтобы
 другие заинтересованные люди могли ознакомиться с ними на русском языке. Более подробно, в этой работе будет проведён разбор со всеми математическими выкладками ряда самых известных зарубежных статей по этой теме
 с дальнейшим объяснением их на более понятном языке. Как результат, курсовая работа будет собой представлять небольшой учебник по нейробайесовским методам.

Помимо вышесказанного, также будет реализован полноценный фреймворк для обучения байесовских нейронных сетей на C++ и протестирован на нескольких наборах данных.
